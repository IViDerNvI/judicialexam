\subsection{物权}

\subsubsection{物权的概念}

\paragraph{概念} 物权是指权利人依法对特定物享有的直接支配和排他的权利。

\subsubsection{所有权}

\paragraph{概念} 指权利人对不动产或动产依法享有的占有、使用、收益和处分的权利。 

\paragraph{所有权的变更} 不动产的所有权变更自登记发生效力,动产的所有权变更自交付发生效力。

\paragraph{业主的建筑物区分所有权} 对建筑物区分所有权的业主,享有对其专有部分的占有、使用、收益和处分的权利,并对建筑物的共有部分享有占有、使用、收益和处分的权利。 

\paragraph{共有} 共有是指两个以上的主体对同一物享有所有权。共有分为按份共有和共同共有. 按份共有者拥有优先购买权.

\paragraph{善意取得} 善意取得是指在无权处分人转让标的物给善意第三人时,善意第三人取得物权的制度。善意取得的条件包括:1. 受让人是善意的;2. 受让人支付了合理对价;3. 标的物已经交付。

\subsubsection{用益物权}

\paragraph{概念} 用益物权是指权利人依法对他人所有的物享有占有、使用和收益的权利,但不包括处分权。

\paragraph{种类} 用益物权包括:1. 地役权;2. 居住权;3. 宅基地使用权权;4. 土地承包经营权;5. 建设用地使用权。

\subsubsection{担保物权}

\paragraph{概念} 担保物权是指债务人或第三人将其所有的物作为债务履行的担保,债权人享有优先受偿的权利。

\paragraph{种类} 担保物权包括:1. 抵押权;2. 留置权;3. 质权。

\paragraph{抵押权} 抵押权是指债务人或第三人将不动产或动产作为债务履行的担保,债权人享有优先受偿的权利。抵押权的设立需要登记。

\paragraph{留置权} 留置权是指债权人对债务人的动产享有的占有和处分的权利,以担保债务的履行。

\paragraph{质权} 质权是指债务人或第三人将动产或权利作为债务履行的担保,债权人享有优先受偿的权利。
