\subsecction{民事法律行为}

\subsubsection{概念}

民事法律行为是指民事主体以设立、变更或消灭民事权利义务为目的的行为。

\subsubsection{效力}

\paragraph{生效要件}

民事法律行为的生效需要满足以下要件:

\begin{itemize}
	\item 行为人具有相应的民事行为能力。
	\item 意思表示真实。
	\item 行为不违反法律规定,不违背公序良俗。
\end{itemize}

\paragraph{无效的民事法律行为}

\begin{itemize}
	\item 无民事行为能力人实施的民事法律行为。
	\item 行为人与相对人以虚假的意思表示实施的民事法律行为。
	\item 行为人与相对人恶意串通,损害他人合法权益的民事法律行为。
	\item 违背公序良俗的民事法律行为。
	\item 违反法律,行政法规的强制性规定的民事法律行为。
\end{itemize}

\paragraph{可撤销的民事法律行为}

\begin{itemize}
	\item 重大误解
	\item 欺诈
	\item 胁迫
	\item 显失公平
\end{itemize}

\paragraph{效力待定的民事法律行为}

\begin{itemize}
	\item 欠缺代理权的代理行为。追认后生效。
	\item 限制民事行为能力人实施的民事法律行为。经法定代理人追认后生效。
	\item 债务承担
\end{itemize}
