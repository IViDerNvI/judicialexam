\subsection{诉讼时效}

\subsubsection{概念} 超过诉讼时效, 消灭权利人的胜诉权.

\paragraph{排除情况} 不适用诉讼时效的请求权.

\begin{itemize}
    \item 请求停止侵害、排除妨碍、消除危险的请求权
    \item 请求支付抚养费、赡养费、扶养费的请求权
    \item 不动产物权和登记的动产物权的权利人请求返还财产
    \item 依法不适用的其他请求权
\end{itemize}

\paragraph{诉讼时效的期间} 一般而言为三年, 但有些特殊情况. 国际货物买卖, 技术进出口的为四年, 人寿保险为五年. 最长为二十年.

\subsubsection{诉讼时效中止}

\paragraph{概念} 诉讼时效最后六个月内, 因不可抗的法定事由暂停计算时效.

\paragraph{中止的事由} 包括:

\begin{itemize}
    \item 不可抗力
    \item 无民事行为能力人或限制民事行为能力人没有法定代理人, 或者法定代理人不能代理的
    \item 继承开始后未确定继承人或遗产管理人的
    \item 权利人被义务人或其他人控制的
    \item 其他导致权利人不能行使请求权的事由
\end{itemize}

\subsection{诉讼时效中断}

\paragraph{概念} 诉讼时效中断是指在诉讼时效期间内, 因某些法定事由的发生, 导致诉讼时效的计算被无效, 并且在中断后重新计算.

\paragraph{中断的事由} 包括:

\begin{itemize}
    \item 权利人向义务人提出请求
    \item 权利人提起诉讼或者申请仲裁
    \item 义务人同意履行义务
    \item 其他法律规定的中断事由
\end{itemize}
