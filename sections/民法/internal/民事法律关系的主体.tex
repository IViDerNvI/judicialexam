\subsection{民事法律关系的主体}

\subsubsection{自然人}

\paragraph{民事权利能力} 指的是受法律确认的自然人享有民事权利的资格。根据《民法典》第九条,自然人自出生时起就享有民事权利能力,死亡时终止。

出生时间和死亡时间以出生证明和死亡证明为准,没有的以户籍登记或者其他有效身份登记记载的时间为准,有其他证据可推翻的除外。

\paragraph{民事行为能力} 指的是自然人能以自己的行为享有民事权利和承担民事义务的资格。

\begin{itemize}
	\item 完全民事行为能力人:18周岁以上,可以独立实施民事法律行为;16周岁以上的,以自己的劳动收入为主要生活来源的。。
	\item 限制民事行为能力人:8周岁以上的未成年人,不能完全辨认自己行为的成年人。可以独立实施纯获利益的或与其智力、精神健康状况相适应的民事法律行为。
	\item 无民事行为能力人:不满8周岁,或者不能辨认自己行为的自然人。
\end{itemize}

\paragraph{监护人} 指的是对无民事行为能力人和限制民事行为能力人进行监护的人。监护人可以是配偶、父母、祖父母、外祖父母、兄弟姐妹、其他亲属,或者经人民法院指定的其他人。监护人不得处分被监护人的财产权利,除了维护被监护人利益的必要行为。

\paragraph{监护}

\paragraph{宣告失踪} 指的是人民法院根据利害关系人申请宣告某人下落不明的法律程序。 要求下落不明满两年后,发出寻找公告三个月后确认的,法院判决宣告失踪。

\paragraph{宣告死亡} 指的是人民法院根据利害关系人申请宣告某人死亡的法律程序。对同一自然人,有宣告失踪和宣告死亡的情况,符合宣告死亡条件的,人民法院应当宣告死亡。

\subsubsection{法人}

\paragraph{营利法人} 股份有限公司、有限责任公司、合伙企业等。

\paragraph{非营利法人} 社会团体、基金会、事业单位等。

\paragraph{特别法人} 指的是法律规定的特殊法人,如农村集体经济组织、城市居民委员会、村民委员会等。

\subsubsection{非法人组织}

\paragraph{非法人组织} 指的是没有法人资格,但能以自己的名义从事民事活动的组织,
