\subsection{国家赔偿}

\begin{quote}
    \href{https://www.gjxfj.gov.cn/gjxfj/xxgk/fgwj/flfg/webinfo/2016/03/1460585589927542.htm}{国家赔偿法原文}
\end{quote}

\subsubsection{概述}

国家赔偿是指国家机关及其工作人员在行使职权时, 给公民, 法人和其他组织造成损害时, 依法给予的经济补偿.

\subsubsection{司法赔偿}

司法赔偿是指司法机关及其工作人员在行使职权时, 给公民, 法人和其他组织造成损害时, 依法给予的经济补偿.

\subsubsection{行政赔偿}

行政赔偿是指行政机关及其工作人员在行使职权时, 给公民, 法人和其他组织造成损害时, 依法给予的经济补偿.

\paragraph{获赔范围} 人身权, 财产权受到侵犯

\subparagraph{人身权遭受侵犯}

\begin{itemize}
    \item 违法拘留或者违法采取限制公民人身自由的行政强制措施的;
    \item 非法拘禁或者以其他方法非法剥夺公民人身自由的;
    \item 以殴打等暴力行为或者唆使他人以殴打等暴力行为造成公民身体伤害或者死亡的;
    \item 违法使用武器、警械造成公民身体伤害或者死亡的;
    \item 造成公民身体伤害或者死亡的其他违法行为。
\end{itemize}

\subparagraph{财产权受到侵犯}

\begin{itemize}
    \item 违法实施罚款、吊销许可证和执照、责令停产停业、没收财物等行政处罚的;
    \item 违法对财产采取查封、扣押、冻结等行政强制措施的;
    \item 违反国家规定征收财物、摊派费用的;
    \item 造成财产损害的其他违法行为。
\end{itemize}

\subparagraph{排除范围}

\begin{itemize}
    \item 行政机关工作人员与行使职权无关的个人行为;
    \item 因公民、法人和其他组织自己的行为致使损害发生的;
    \item 法律规定的其他情形。
\end{itemize}

\paragraph{赔偿义务机关}

\begin{itemize}
    \item 行政机关及其工作人员行使行政职权侵犯公民、法人和其他组织的合法权益造成损害的,该行政机关为赔偿义务机关。
    \item 两个以上行政机关共同行使行政职权时侵犯公民、法人和其他组织的合法权益造成损害的,共同行使行政职权的行政机关为共同赔偿义务机关。
    \item 法律、法规授权的组织在行使授予的行政权力时侵犯公民、法人和其他组织的合法权益造成损害的,被授权的组织为赔偿义务机关。
    \item 受行政机关委托的组织或者个人在行使受委托的行政权力时侵犯公民、法人和其他组织的合法权益造成损害的,委托的行政机关为赔偿义务机关。
    \item 赔偿义务机关被撤销的,继续行使其职权的行政机关为赔偿义务机关;没有继续行使其职权的行政机关的,撤销该赔偿义务机关的行政机关为赔偿义务机关。
    \item 经复议机关复议的,最初造成侵权行为的行政机关为赔偿义务机关,但复议机关的复议决定加重损害的,复议机关对加重的部分履行赔偿义务。
    \item 赔偿义务机关赔偿损失后,应当责令有故意或者重大过失的工作人员或者受委托的组织或者个人承担部分或者全部赔偿费用。
\end{itemize}

\subsubsection{赔偿方式} 国家赔偿以支付赔偿金为主要方式。

能够返还财产或者恢复原状的,予以返还财产或者恢复原状。

赔偿请求人要求国家赔偿的,赔偿义务机关、复议机关和人民法院不得向赔偿请求人收取任何费用。对赔偿请求人取得的赔偿金不予征税。



