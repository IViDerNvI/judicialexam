\subsection{行政诉讼}

\begin{quote}
	\href{https://www.gov.cn/flfg/2006-10/29/content_1499268.htm}{行政诉讼法原文}
\end{quote}

\subsubsection{行政诉讼的概念} 民告官

\subsubsection{行政诉讼的基本原则}

\paragraph{依法受理}

\paragraph{法院独立行使审判权}

\paragraph{事实为依据, 法律为准绳}

\paragraph{行政行为合法性审查}

\paragraph{当事人平等}

\paragraph{本民族语言使用}

\paragraph{当事人有权辩论}

\paragraph{检察院的监督}

\subsubsection{行政诉讼的基本制度}

\paragraph{合议制} 由审判员组成或者由审判员和陪审员组成的合议庭进行审理. 保证3人以上的单数.

\paragraph{回避制度} 审判人员, 书记员, 翻译人员, 鉴定人, 勘验人.

\paragraph{公开审判} 除国家秘密, 个人隐私外, 行政诉讼应当公开审理. 商业秘密申请后可不公开.

\paragraph{二审终审制}

\subsubsection{行政诉讼的受案范围}

\paragraph{行政诉讼的排除范围}

\begin{itemize}
	\item 国防、外交等国家行为;
	\item 行政法规、规章或者行政机关制定、发布的具有普遍约束力的决定、命令;
	\item 行政机关对行政机关工作人员的奖惩、任免等决定;
	\item 法律规定由行政机关最终裁决的具体行政行为。
\end{itemize}

\subsubsection{行政诉讼的管辖}

\paragraph{级别管辖}

\subparagraph{基层人民法院} 普通一审

\subparagraph{中级人民法院} 专业性强的海关相关案件; 级别在国务院部门, 县级以上地方政府的; 本辖区内重大复杂的案件; 其他案件

\subparagraph{高级人民法院} 本辖区内重大复杂的案件

\subparagraph{最高人民法院} 全国范围内重大复杂的案件

\paragraph{地域管辖}

\subparagraph{一般} 原告就被告; 原机关地或者复议机关地

\subparagraph{特殊} 不动产相关案件在不动产所在地, 限制人身自由可在原告地或被告地

\subsubsection{诉讼当事人}

\paragraph{原告} 认为行政机关的具体行政行为侵犯其合法权益的公民、法人或者其他组织。公民死亡情况下, 近亲属以自己名义提起诉讼.

\paragraph{被告} 行政机关或者法律、法规授权的组织。

\subparagraph{一般被告} 行政机关

\subparagraph{经过复议} 维持决定下行政机关和复议机关为共同被告; 改变决定下行政机关和复议机关为单独被告; 撤销决定下复议机关为被告; 未作出决定下可分别起诉。

\subparagraph{两个以上行政机关共同作出具体行政行为} 由作出具体行政行为的行政机关共同作为被告。

\subparagraph{委托组织做的具体行政行为} 由委托组织和委托机关共同作为被告。

\subparagraph{被撤销的行政机关} 由继续行使其职权的行政机关作为被告。

\subsubsection{行政诉讼证据}

\paragraph{证据的种类}

\begin{itemize}
	\item 书证;
	\item 物证;
	\item 视听资料;
	\item 证人证言;
	\item 当事人的陈述;
	\item 鉴定结论;
	\item 勘验笔录、现场笔录。
\end{itemize}

\paragraph{举证责任} 被告对作出的具体行政行为负有举证责任; 诉讼过程中, 被告不得自行向原告和证人收集证据.

例外情况有两种. 被告不作为的, 向被告提出申请的证据; 行政赔偿, 补偿的案件, 造成损害的案件;

\subsubsection{行政诉讼程序}

\paragraph{起诉} 收到复议决定书的15日内提起诉讼; 直接向法院起诉的, 收到具体行政行为之日起6个月内提起诉讼。

\paragraph{受理} 公开审理原则; 不停止执行原则; 不适用调解原则

\paragraph{审理}

\subparagraph{一审} 应当在立案之日起6个月内判决; 一律公开宣判

\subparagraph{二审} 上诉在判决书送达之日起15日内, 在裁定书送达之日起10日内上诉; 二审应当在立案之日起3个月内判决

\subparagraph{再审} 分为当事人申请再审, 检察院抗诉和法院依职权启动
