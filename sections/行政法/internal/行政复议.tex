\subsection{行政复议}

\begin{quote}
    \href{https://www.gov.cn/yaowen/liebiao/202309/content_6901584.htm}{行政复议法原文}
\end{quote}

\subsubsection{行政复议的概念} 公民, 法人或者其他组织认为行政主体的具体行政行为侵犯其合法权益, 依法向行政复议机关提出复查申请. 行政复议机关就具体行政行为进行合法性, 适当性审查.

\subsubsection{行政复议的受案范围} 包括影响行政相对人权利的具体行政行为和附带审查部分抽象行政行为

\paragraph{影响行政相对人权利的具体行政行为}

\begin{itemize}
    \item 对行政机关作出的行政处罚决定不服;
    \item 对行政机关作出的行政强制措施、行政强制执行决定不服;
    \item 申请行政许可,行政机关拒绝或者在法定期限内不予答复,或者对行政机关作出的有关行政许可的其他决定不服;
    \item 对行政机关作出的确认自然资源的所有权或者使用权的决定不服;
    \item 对行政机关作出的征收征用决定及其补偿决定不服;
    \item 对行政机关作出的赔偿决定或者不予赔偿决定不服;
    \item 对行政机关作出的不予受理工伤认定申请的决定或者工伤认定结论不服;
    \item 认为行政机关侵犯其经营自主权或者农村土地承包经营权、农村土地经营权;
    \item 认为行政机关滥用行政权力排除或者限制竞争;
    \item 认为行政机关违法集资、摊派费用或者违法要求履行其他义务;
    \item 申请行政机关履行保护人身权利、财产权利、受教育权利等合法权益的法定职责,行政机关拒绝履行、未依法履行或者不予答复;
    \item 申请行政机关依法给付抚恤金、社会保险待遇或者最低生活保障等社会保障,行政机关没有依法给付;
    \item 认为行政机关不依法订立、不依法履行、未按照约定履行或者违法变更、解除政府特许经营协议、土地房屋征收补偿协议等行政协议;
    \item 认为行政机关在政府信息公开工作中侵犯其合法权益;
    \item 认为行政机关的其他行政行为侵犯其合法权益。
\end{itemize}

\paragraph{附带审查部分抽象行政行为}

\begin{itemize}
    \item 国务院部门的规范性文件;
    \item 县级以上地方各级人民政府及其工作部门的规范性文件;
    \item 乡、镇人民政府的规范性文件;
    \item 法律、法规、规章授权的组织的规范性文件。
\end{itemize}

\paragraph{排除范围}

\begin{itemize}
    \item 国防、外交等国家行为;
    \item 行政法规、规章或者行政机关制定、发布的具有普遍约束力的决定、命令等规范性文件;
    \item 行政机关对行政机关工作人员的奖惩、任免等决定;
    \item 行政机关对民事纠纷作出的调解。
\end{itemize}

\subsubsection{复议机关的确定} 

\paragraph{省级以下政府} 向上一级人民政府

\paragraph{地方政府部门} 同级政府或上一级主管部门

\paragraph{垂直领导机关} 向上一级主管部门

\paragraph{省部级单位} 向原机关自己

\paragraph{政府派出机关} 设立该机关的人民政府

\paragraph{政府派出机构} 向派出该机构的主管部门或该主管部门的同级政府

\paragraph{被授权组织} 直接管理该组织的机关

\paragraph{多个行政机关} 其共同上级机关

\paragraph{被撤销的机关} 继续行使其职权机关的上一级机关 

\subsubsection{行政复议的程序}

\paragraph{期限} 公民、法人或者其他组织认为行政行为侵犯其合法权益的,可以自知道或者应当知道该行政行为之日起六十日内提出行政复议申请;但是法律规定的申请期限超过六十日的除外。因不可抗力或者其他正当理由耽误法定申请期限的,申请期限自障碍消除之日起继续计算。

\paragraph{方式} 申请人申请行政复议,可以书面申请;书面申请有困难的,也可以口头申请。

\paragraph{审理} 书面复议, 不停止执行, 被申请人负有举证责任

\paragraph{复议决定} 受理之日起60日做出, 但法律规定的除外.

