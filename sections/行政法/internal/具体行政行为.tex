\subsection{具体行政行为}

\subsubsection{行政处罚}

\paragraph{行政处罚的概念} 行政机关对公民, 法人或其他组织做出的, 危害国家行政管理秩序的行为, 依法作出的, 以减损权益或增加义务的方式予以惩戒的行为

\paragraph{行政处罚的种类} 

\subparagraph{警告, 通告批评}

\subparagraph{罚款, 没收违法所得, 没收非法财物}

\subparagraph{暂扣许可证件, 降低资质等级, 吊销许可证件}

\subparagraph{限制开展生产经营活动, 责令停产停业, 责令关闭, 限制从业}

\subparagraph{行政拘留}

\subparagraph{法律, 行政法规规定的其他行政处罚}

\paragraph{行政处罚的设定}

\subparagraph{法律} 法律可以设定各种行政处罚

\subparagraph{行政法规} 行政法规可以设定行政处罚, 但不得设定限制人身自由的行政处罚

\subparagraph{地方性法规} 地方性法规可以设定行政处罚, 但不得设定限制人身自由和吊销营业执照的行政处罚

\subparagraph{规章} 尚未制定法律的, 规章可以设定警告, 通报批评或者一定数量罚款的行政处罚

\paragraph{行政处罚的适用}

\subparagraph{管辖} 行政处罚由违法行为发生地的县级以上人民政府具有行政处罚权的行政机关管辖

\subparagraph{裁量情节}

\begin{itemize}
    \item 不予处罚的: 不满14周岁的人; 精神病人, 智力残疾人在不能辨认或者不能控制自己行为时有违法行为的; 违法行为轻微并及时纠正没有危害后果的; 当事人有证据证明没有主观过错的; 初次违法且情节轻微的.
    \item 从轻处罚的: 以满14周岁不满16周岁的人; 主动消除或减轻危害后果的; 受他人胁迫或诱骗实施违法行为的; 主动供述行政机关尚未掌握的违法行为的; 有立功表现的.
\end{itemize}

\subparagraph{追究时效} 行政违法行为自终止之日算起, 在2年内没发现的, 不再予以处罚; 涉及公民生命安全, 金融安全且有危害后果的, 延长至5年.

\subparagraph{一事不再罚} 同一违法行为, 不得给予两次以上的罚款处罚

\paragraph{行政处罚的程序}

\subparagraph{简易程序} 满足如下条件: 违法事实确凿, 由法定依据, 较小数额罚款或警告的行政处罚. (数额较小指公民200元以下, 法人组织300元以下)

\subparagraph{一般程序} 一般程序包括三个步骤, 即调查, 决定和送达.

\begin{itemize}
    \item 调查: 行政机关查明当事人违法事实的过程, 行政机关派出执法人员不少于2人, 并且应当告知当事人拟作出的行政处罚决定, 向当事人说明事实和理由, 听取当事人的陈述和申辩意见, 不得因为当事人申辩而加重处罚.
    \item 决定: 行政机关查明违法事实, 依据法律作出处罚.
    \item 送达: 行政处罚决定书应在宣告后当场送达当事人, 当事人不在场的, 应当在7日内送达.
\end{itemize}

\subparagraph{听证程序}

\begin{itemize}
    \item 举行听证会的条件: 第一, 行政机关拟作出较大数额罚款, 没收较大数额违法所得, 没收较大价值非法财物, 降低资质等级, 吊销许可证件, 暂扣许可证件, 责令停产停业, 责令关闭, 限制从业资格等行政处罚; 第二, 当事人提出听证要求的.
    \item 听证会程序
    \begin{itemize}
        \item 听证时间: 当事人提出听证的, 应当在行政机关告知后5日内提出; 行政机关在举行听证会7日前通知当事人听证时间和地点.
        \item 听证方式: 听证公开举行, 涉及国家秘密, 商业秘密和个人隐私的除外.
        \item 听证主持: 由行政机关指定的非本案调查人员主持, 当事人认为其有利害关系的, 有权要求回避.
        \item 听证笔录: 听证会应当制作笔录, 行政机关应当根据笔录做出处罚决定
    \end{itemize} 
\end{itemize}

\paragraph{行政处罚的执行}

\subparagraph{罚缴分离} 罚款决定的行政机关与收缴罚款的行政机关应当分开; 除了当场收缴的情形外, 行政机关不得自行收缴罚款.

\subparagraph{当场收缴} 行政机关可以当场收缴罚款, 但必须满足: 依法给予100元以下罚款的或不当场收缴事后难以执行的. 在交通不便地区, 由当事人提出, 行政机关及其执法人员可以当场收缴罚款. 当场收缴须满足一定程序. 第一, 行政机关应当向当事人出示执法证件; 第二, 行政机关应当向当事人说明违法事实和处罚依据; 第三, 行政机关应当向当事人说明罚款数额; 第四, 行政机关及其执法人员必须向当事人出具财政部门制发的专用票据, 不出具的情况, 当事人可以拒绝缴纳. 第二, 当场缴纳的罚款, 应当自缴纳之日起两日内交至行政机关, 行政机关在两日内将罚款交至指定的银行账户.

\subsubsection{行政强制}

\paragraph{行政强制的概念} 行政强制包括行政强制措施和行政强制执行. 行政强制措施是指行政机关在行政管理过程中, 为制止违法行为, 防止证据损毁, 避免危害发生, 控制危险扩大等情形, 依法对公民的人身自由实施暂时性限制, 或者对公民, 法人或者其他组织的财物实施暂时性控制的行为. 行政强制执行是指行政机关或者行政机关申请人民法院, 对不履行行政决定的公民, 法人或者其他组织, 依法强制履行义务的行为。

\paragraph{行政强制的种类} 行政强制措施包括限制公民人身自由, 查封场所、设施或者财物, 扣押财物, 冻结存款、汇款, 其他行政强制措施. 行政强制执行包括: 加处罚款或者滞纳金, 划拨存款、汇款, 拍卖或者依法处理查封、扣押的场所、设施或者财物, 排除妨碍、恢复原状, 代履行, 其他强制执行方式.

\subsubsection{行政许可}

\paragraph{行政许可的概念} 行政许可是指行政机关根据公民, 法人或者其他组织的申请, 经依法审查, 准予其从事特定活动的行为. 主要包括:

\begin{itemize}
    \item 直接涉及国家安全、公共安全、经济宏观调控、生态环境保护以及直接关系人身健康、生命财产安全等特定活动,需要按照法定条件予以批准的事项;
    \item 有限自然资源开发利用、公共资源配置以及直接关系公共利益的特定行业的市场准入等,需要赋予特定权利的事项;
    \item 提供公众服务并且直接关系公共利益的职业、行业,需要确定具备特殊信誉、特殊条件或者特殊技能等资格、资质的事项;
    \item 直接关系公共安全、人身健康、生命财产安全的重要设备、设施、产品、物品,需要按照技术标准、技术规范,通过检验、检测、检疫等方式进行审定的事项;
    \item 企业或者其他组织的设立等,需要确定主体资格的事项;
    \item 法律、行政法规规定可以设定行政许可的其他事项。
\end{itemize}

\paragraph{行政许可的种类} 包括普通许可, 特许, 认可, 核准, 登记.

\subsubsection{其他具体行政行为} 包括行政征收, 行政确认, 行政给付, 行政奖励和行政裁决.