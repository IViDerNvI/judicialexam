\subsection{概述}

\subsubsection{行政法}

\paragraph{行政法} 行政法调整的是不平等主体的法律关系

\subsubsection{行政法基本原则}

\paragraph{基本原则} 合法行政原则, 合理行政原则, 程序正当原则, 高效便民原则, 诚实守信原则, 责权统一原则.

\subparagraph{合法行政原则} 包括对现行法律的遵守和依照法律授权活动.

\subparagraph{合理行政原则} 指行政决定应具有理性. 包括比例原则, 即在实施行政行为时, 如须在多项方案中选择, 要选取对行政相对人权益损害最小的方案.

\subparagraph{程序正当原则} 包括行政公开, 公众参与和回避.

\subparagraph{高效便民原则} 包括行政效率和便利当事人原则.

\subparagraph{诚实守信原则} 包括行政信息真实和保护公民信赖利益原则.

\subparagraph{责权统一原则} 指法律赋予行政机关权利同时赋予了其责任和义务.

\subsubsection{行政主体与行政行为} 

\paragraph{行政主体} 包括行政机关和法律法规授权的组织.

\subparagraph{行政机关} 行政机关包括政府及职能部门, 派出机关和派出机构. 

\begin{itemize}
    \item 政府: 指行政区级别的政府
    \item 政府职能部门: 政府中具体负责某项事务的部门
    \item 派出机关: 包括行政公署, 区公所, 街道办事处, 分别由省级, 县级, 市级人民政府设立.
    \item 派出机构: 由有权地方人民政府的职能部门派出的
\end{itemize}

\subparagraph{被授权组织} 一般是国有事业单位和企业单位.

\paragraph{行政行为} 

\subparagraph{行政行为的效力}

\begin{itemize}
    \item 公定力: 行政行为一经作出被推定合法.
    \item 确认力: 不得随意更改撤销.
    \item 拘束力: 相关人员必须遵守服从
    \item 执行力: 行政主体依法采取措施使行政行为的内容实现
\end{itemize}

\subparagraph{行政行为的分类}

\begin{itemize}
    \item 抽象与具体: 行政行为对象具体与否
    \item 羁束和自由裁量: 行政行为受法律约束的程度
    \item 依职权的和依申请的: 行政主体主动做出行政行为的能力
    \item 内部和外部: 行政行为作用对象
\end{itemize}



